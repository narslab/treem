\documentclass[preprint, 11pt]{elsarticle}
%\usepackage{geometry}
%\geometry{top=1in, left=1.3in,right=1.3in,bottom=1in}

\usepackage{booktabs,calc,soul,xcolor}
\setstcolor{red}

\usepackage{marginnote} 
% Marginpar width
%Marginpar width
\newcommand{\pts}[1]{\marginpar{ \small\hspace{0pt} \textit{[#1]} } }
\setlength{\marginparwidth}{1.1in}


%\reversemarginpar
%\setlength{\marginparsep}{.02in}

%% Fonts
% \usepackage{fourier}
% \usepackage[T1]{pbsi}

\usepackage{lmodern}
\usepackage[T1]{fontenc}
%\usepackage{minted}

\usepackage{rotating}
\usepackage{amsmath, amssymb}
%% Cite Title
% \usepackage[style=numeric,backend=biber,sorting=none,maxcitenames=2,maxbibnames=99,doi=false,isbn=false,url=false,eprint=false]{biblatex}
% \addbibresource{TREEM.bib}

%\usepackage{lineno}
%\renewcommand\linenumberfont{\normalfont\small}
%\linenumbers



\usepackage{graphicx}
\graphicspath{{./images/}}
 
\usepackage{rotating}
\usepackage{longtable}

%\usepackage{enumerate}
\usepackage{paralist}
\usepackage{epstopdf,subfigure,hyperref,enumerate,polynom,polynomial}
\usepackage{multirow,minitoc,fancybox,array,multicol}

% \definecolor{slblue}{rgb}{0,.3,.62}
% \hypersetup{
%     colorlinks,%
%     citecolor=blue,%
%     filecolor=blue,%
%     linkcolor=blue,
%     urlcolor=slblue
% }


%%%TIKZ
\usepackage{tikz}
\usepackage{pgfplots}
\usepackage{pgfplotstable}
\usepackage{pgfgantt}
\pgfplotsset{compat=newest}

\usetikzlibrary{arrows,shapes,positioning,shapes.geometric}
\usetikzlibrary{decorations.markings}
\usetikzlibrary{shadows,automata}
\usetikzlibrary{patterns}
\usetikzlibrary{trees,mindmap,backgrounds}
%\usetikzlibrary{circuits.ee.IEC}
\usetikzlibrary{decorations.text}
\usetikzlibrary{ decorations.pathreplacing,decorations.pathmorphing}
\tikzset{no shadows/.style={general shadow/.style=}}

% Default fixed font does not support bold face
\DeclareFixedFont{\ttb}{T1}{txtt}{bx}{n}{8} % for bold
\DeclareFixedFont{\ttm}{T1}{txtt}{m}{n}{8}  % for normal


\newcommand{\osn}{\oldstylenums}
\newcommand{\dg}{^{\circ}}
\newcommand{\lt}{\left}
\newcommand{\rt}{\right}
\newcommand{\pt}{\phantom}
\newcommand{\tf}{\therefore}
\newcommand{\?}{\stackrel{?}{=}}
\newcommand{\fr}{\frac}
\newcommand{\dfr}{\dfrac}
\newcommand{\tn}{\tabularnewline}
\newcommand{\nl}{\newline}
\newcommand\relph[1]{\mathrel{\phantom{#1}}}
\newcommand{\cm}{\checkmark}
\newcommand{\ol}{\overline}

%%% COLORS %%%%
\newcommand{\rd}{\color{red}}
\newcommand{\bl}{\color{blue}}
\newcommand{\pl}{\color{purple}}
\newcommand{\og}{\color{orange!90!black}}
\newcommand{\gr}{\color{green!40!black}}
% Custom colors
%\usepackage{color}
\definecolor{deepblue}{rgb}{0,0,0.5}
\definecolor{deepred}{rgb}{0.6,0,0}
\definecolor{deepgreen}{rgb}{0,0.5,0}


%%% FORMATTING MACROS %%%%%
\newcommand{\nin}{\noindent}
\newcommand{\la}{\lambda}
\renewcommand{\th}{\theta}
\newcommand{\al}{\alpha}
\newcommand{\G}{\Gamma}
\newcommand*\circled[1]{\tikz[baseline=(char.base)]{
    \node[shape=circle,draw,thick,inner sep=1pt] (char) {\small #1};}}

\newcommand{\bc}{\begin{compactenum}[\quad--]}
\newcommand{\ec}{\end{compactenum}}

\newcommand{\p}{\partial}
\newcommand{\pd}[2]{\frac{\partial{#1}}{\partial{#2}}}
\newcommand{\dpd}[2]{\dfrac{\partial{#1}}{\partial{#2}}}
\newcommand{\pdd}[2]{\frac{\partial^2{#1}}{\partial{#2}^2}}

\journal{Transportation Research Interdisciplinary Perspectives }

\begin{document}
 
 
\begin{frontmatter}
  
\title{Impacts of COVID-19 on urban rail transit energy consumption: A case study of Boston}
\author[umass]{Author 1}
\author[umass]{Author 2} 
\author[umass]{Author 3}
\author[umass]{Author 4}


\address[umass]{Department of Civil and Environmental Engineering, University of Massachusetts Amherst, MA 01003, United States}

\begin{abstract}
\end{abstract}

\begin{keyword}
  COVID-19 \sep public transportation \sep energy
\end{keyword}

\end{frontmatter}

\section{Introduction}
{\it Impacts of COVID-19 on transit}

\section{Background}
\subsection{Importance of transit}

\subsection{Energy modeling of rail transit}

\subsection{Environmental impacts of urban rail transit}

\section{Data and methods}
\subsection{Description of Boston network}

\subsection{Summary of data sources and variables}
{\it Energy, ridership, location}

\subsection{Data extraction and cleaning}
{\it Describe processing and trajectory computation; corrections}

Data on train locations and tap-in ridership were extracted from the MBTA database.
The location data (latitude and longitude) were recorded for time intervals of varying duration. Based on the raw data, we use the Haversine formula to calculate the distance between each time record for tracking the train trajectory. And then we can calculate the speed,acceleration accordingly. However after the calculation, we observed many excessive speed values. Then we designed the following algorithm for correction:
\item{Finish the first calculation iteration.}
\item{Find out the excessive speed rows and save them to a separate table for diagnostics analysis.}
\item{Use the excessive speed index to locate the wrong speed row in the original table and replace them by the correct value before}
\item{Correct the vehicle distance and acceleration accordingly.}


\begin{table}[]\footnotesize
    \centering
    \begin{tabular}{l l}\toprule
        \bf Variable & \bf Description \\
       Speed  &  \\\bottomrule
    \end{tabular}
    \caption{Summary of variables used in model}
    \label{tab:my_label}
\end{table}

\subsection{Energy consumption modeling}
A backward model for electric train energy was developed by \cite{wang2017electric}.

\section{Results}

\section{Discussion}

\section{Conclusion}

\section*{Acknowledgments}

\bibliographystyle{elsarticle-num}
\bibliography{../TREEM.bib}

\end{document}

%%% Local Variables:
%%% mode: latex
%%% TeX-master: t
%%% End:

